\documentclass[a4paper]{article}
\usepackage[utf8]{inputenc}
\usepackage[T1]{fontenc}
\usepackage[brazilian]{babel}
\usepackage[hidelinks]{hyperref}

\title{Sistema Helena}
\author{}
\date{}

\begin{document}

\maketitle
\tableofcontents

\newpage

\section{Visão Geral}

Sistema gerenciará os procedimentos no NPJ. Nele (sistema) serão
cadastrados, principalmente, \textit{usuários}, \textit{pessoas},
\textit{processos} e \textit{comentários} nos processos, que
descreverão seu andamento. Outros itens serão cadastrados, mas de
forma auxiliar.

Ao chegar no NPJ, o assistido irá para a recepção, onde o
recepcionista o cadastrará, caso já não esteja cadastrado. O
recepcionista, para o sistema, é um \textit{usuário}; o assistido,
para o sistema, é uma \textit{pessoa}. Após cadastrar o assistido, o
recepcionista poderá cadastrar seu processo. É necessário cadastrar a
pessoa antes de cadastrar seu processo porque, no cadastro do
processo, escolheremos os assistidos do processo (que são pessoas).

Neste momento, o assistido estará cadastrado e haverá para ele um
processo cadastrado, mas não completamente preenchido.  O assistido
entrará na lista de espera para ser atendimento por um estagiário. O
assistido será informado do número de seu processo.

Quando sua vez chegar de ser atendido, o assistido irá até o
estagiário e lhe informará seu número de processo. O estagiário (para
o sistema, um usuário) acessará no sistema a listagem de processos e
achará o processo específico do assistido; editará esse processo se
quiser adicionar detalhes; adicionará um andamento (comentário).

\section{Usuários}

Usuários entram no sistema, adicionam pessoas, processos etc. Podem
ter um ou mais papéis: administrador, coordenador, professor,
recepcionista, estagiário etc. Ainda não definimos exatamente que
papéis fazem o que dentro do sistema, mas podemos fazer isso depois.

Haverá uma tela de cadastro de usuários.  Haverá uma tela de listagem
de usuários. Ao clicar num dos usuários da listagem, sistema mostrará
detalhes do usuário clicado.

\section{Pessoas}

Pessoas figuram nos processos como assistido, requerente etc. São
cadastradas pelos usuários.

Haverá uma tela de cadastro de pessoas.  Haverá uma tela de listagem
de pessoas. Ao clicar numa das pessoas, sistema mostrará detalhes da
pessoa clicada.

\section{Processos}

Um processo, no sistema, pode ou não representar um processo
judicial. Um processo, no sistema, terá campos como:

\begin{itemize}
\item assistidos
\item representantes legais
\item requeridos
\item responsáveis
\item turma
\item tipo de processo
\item descrição
\item data de criação
\item outros
\end{itemize}

Haverá uma tela de listagem de processos. Nesta tela haverá uma opção
de busca de processos. Cada processo que aparece na listagem pode ser
clicado. Ao clicar num processo, o usuário irá para a tela de detalhes
desse processo específico. Na tela de detalhes, verá os detalhes do
processo e seus andamentos (próx. seção). Na tela de detalhes poderá
também acessar a tela de edição do processo e de adição de comentário
(próx. seção).

\subsection{Comentários}

Um usuário poderá adicionar comentários a um processo. Exemplo:
assistido retornou ao NPJ para entregar documentos, mas esqueceu um
deles; isso poderia gerar um comentário como ``assistido entregou
documentos X e Y, no entanto, esqueceu o documento Z''.

\subsection{Lembretes de Documentos}

No atendimento a um assistido, o usuário deve solicitar-lhe uma série
de documentos, dependendo do tipo de processo escolhido. Assim, um
tipo de processo terá uma lista de documentos, e quando este tipo de
processo for escolhido na tela de abertura de processos, a lista de
documentos do tipo deverá ser inserida na tela para lembrar o usuário
de pedir esses documentos ao assistido.

Talvez fazer essa lista em forma de lista com \textit{checkboxes}. O
usuário usaria os \textit{checkboxes} para informar ao sistema que
documento foi entregue ou não.

\subsection{Lembretes de Prazo por Email}
Larissa falou sobre o sistema enviar emails no caso de alguns prazos
estarem perto de vencer. Não lembro as situações exatas. Perguntar
para ela.

\section{Arquivos}
Um processo vai poder ter arquivos anexados a ele, e os comentários de
um processo também poderão ter arquivos anexados a eles. Antes de
implementar essa funcionalidade, pensar como simplificá-la.

\section{Meus Processos}
Haverá um item no menu principal chamado \textit{Meus Processos}. Ao ser
clicado, levará a uma tela que mostrará só os processos do usuário logado,
considerando os processos em que seja responsável.

\end{document}
